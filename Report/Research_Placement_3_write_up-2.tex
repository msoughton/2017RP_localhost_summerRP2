\documentclass[a4paper,12,oneside,notitlepage]{report}
\usepackage[letterpaper, portrait, margin=2cm]{geometry}
\usepackage{listings}
\usepackage{url}
\usepackage{mhchem}
\usepackage{graphicx}
\usepackage{caption}
\usepackage{amsmath}
\usepackage{fancyhdr}
\usepackage{siunitx}
\usepackage{float}
\usepackage{epsfig,amsmath,amssymb,rotating,array}
\pagestyle{fancy}
\lhead{University of Sussex}
\rhead{Candidate No. 133884}
\renewcommand{\headrulewidth}{0.4pt}
\renewcommand{\footrulewidth}{0.4pt}
%\renewcommand{\section}[2]{}%
\renewcommand{\chapter}[2]{}%

\begin{document}
\title{Summer Research Placement Report: Supernovae Neutrinos}
\author{Michael Soughton, ms731@sussex.ac.uk \\
Supervisor: Simon Peeters,  \\
Working with: Bruno Zamorano}
\date{21$^{st}$ September, 2017}
\maketitle
\providecommand{\e}[1]{\ensuremath{\times 10^{#1}}}


\vspace{14cm}
\begin{abstract}
\noindent
\begin{center}
The Deep Underground Neutrino Experiment (DUNE), which is part of the The Long-Baseline Neutrino Facility (LBNF) currently under construction, aims to provide new information regarding the production of neutrinos inside supernovae. First, a reliable trigger for supernovae neutrinos must be made so that the full data-readout of a supernovae can be saved (full data steam is normally discarded). This project looked into obtaining useful Monte Carlo truth information of supernovae neutrinos which could be compared to background noise in order to produce a trigger. Initial findings show that hits from supernovae neutrinos occur within the detector in sufficiently different ways to background radiological noise such that the two sources could be decomposed to a good extent.
\end{center}
\end{abstract}

\newpage
\section*{\fontsize{11}{11}\selectfont Introduction }
During the last decade, the neutrino community has made efforts to construct a new generation of neutrino detectors. The Long-Baseline Neutrino Facility (LBNF) and the Deep Underground Neutrino Experiment (DUNE), due to be constructed in the next decade, will aim to have a 1.2 MW muon neutrino beam constructed at Fermilab by 2026 which will be upgraded to 2.4 MW by 2030. A near detector will detect neutrinos from the beam near it's source and the DUNE far detector will detect beam neutrinos and supernova neutrinos as well as other background neutrinos eight-hundred miles away and will consist of four 10 kt Liquid Argon Time Projection Chamber (LArTPC) modules located deep underground. DUNE will search for CP-violation in neutrino oscillations, determine the ordering of the neutrino masses, test the three-neutrino paradigm, search from proton decay if it exits and will provide new information on how supernovae explode \cite{LBNEVol1} and what new physics can be learnt from a supernova neutrino burst - there has only been one recorded supernova neutrino event. DUNE should be able to determine the time, flavour and energy structure of a neutrino burst, however in order to retrieve useful data from a neutrino burst, we must develop a trigger so that the data readout (4.6 Tb s$^{-1}$ from four 10 kt LArTPCs) will only be saved to memory in the event of a supernova. My project looked at investigating the properties of neutrinos and the resulting 'hits' within a detector from a supernova neutrino burst and what we will need to know to be able to trigger a supernova neutrino readout to aquire useful data should one occur. 
\vspace{0.5cm}

\section*{\fontsize{11}{11}\selectfont Supernovae Neutrinos, Liquid Argon Time Proportion Chambers (LArTPCs) and Monte Carlo (MC) Simulations }
Since my project would look into the detection of supernovae neutrinos, an understanding of the origin of these neutrinos and how they will be detected was required. Supernovae produce a very large number of neutrinos - around 99\% of their gravitational binding energy is converted into neutrinos. Supernovae are of two main types, type I and type II, each with sub-categories. Type I supernovae occur when a white dwarf - a lower mass star in its final evolutionary stage, which is prevented from collapsing from the pressure of electron degeneracy, accretes enough matter from another nearby star that the star becomes massive enough that it is favourable (it requires less energy) for electrons to be captured by protons that it does to fill electron states (electrons follow Fermi-Dirac statistics) and so neutrinos and neutrons are produced by electron capture as
$$e^- + p \rightarrow \nu_e + n$$
and the star starts to collapse until collapse is halted by stronger neutron degeneracy. The mass at which a white dwarf will collapse is always 1.44 Solar Masses (the Chandreska Limit). Type II supernovae occur when a supermassive star has reached the point in it's lifetime when it has converted a significant amount of its matter into heavier elements. Fusing elements heavier than iron is an endothermic process, so the star does not produce enough energy in the core to prevent collapse. In both cases, as the star undergoes core collapse, there are a large number of electrons and protons with enough energy to undergo inverse beta-decay/electron capture. This phase lasts for about 10 ms \cite{Scholberg}. As there is so much in-falling matter, many of the neutrinos interact with it, even though neutrinos cross-section is very small. The neutrinos that escape can be seen as an initial neutrino burst, which is when the detector must trigger a supernova. Most electron flavours are of electron neutrino, then of anti-electron neutrino with few (anti)-muon and (anti)-tau neutrinos.
\vspace{0.5cm}
\\It is thought that the interaction of these neutrinos with the in-falling matter is enough to prevent all the matter from falling into a black hole and to launch the matter outwards. After the accretion phase, there is the cooling phase as the clouds of matter expand outwards. We still see neutrinos on the order of 10 s, but the luminosity of all flavours will gradually decline. There will be minimal neutrino oscillations in the vacuum of space, although if the supernova occurs on the other side of the Earth to the detector, then there will be neutrino oscillations due to the Earth. The oscillations through the Earth are fairly well known down to a certain depth, although oscillations through the center will introduce uncertainty unless there is a detector on the other side of the Earth to give un-shifted results. 
\vspace{0.5cm}
\\A LArTCP is a large chamber mostly filled with Argon-40. Electron neutrinos will interact with the Argon-40 through inverse beta decay through the interaction
$$\nu_e + \ce{^{40}Ar} \rightarrow e^- +\ce{^{40}K^*} +m\gamma$$
where and $m$ is a positive integer (usually less than or equal to three). Similarly, anti neutrinos interact through
$$\bar{\nu_e}+\ce{^{40}Ar} \rightarrow e^+ + \ce{^{40}Cl^*}+m \gamma$$
LArTPCs only interact with supernova muon and tau neutrinos through NC (Neutral Current - a neutral $Z$ Boson mediates the interaction between neutrino and electron scattering)interactions as they do not have enough energy to produce a muon or a tau electron unlike the beam muon (and anti) neutrinos which have much more energy and so can interact through CC (Charged Current - a charged $W^+$ or $W^-$ Boson mediates a neutrino going to a lepton and a lepton going to a neutrino simultaneously) interactions.
\vspace{0.5cm}
\\The produced lepton travels off some trajectory with a high momentum, meaning that it ionises more Argon-40 along the way. An electric field (of strength 500 Vcm$^{-1}$) is applied across the TPC by an Anode Plane Assembly (APA) at one end and a cathode plane at the other \cite{LBNEVol4}. The ionised Argon will move towards the cathode plane and the electrons towards to APA. The APA consists of three planes of wires as shown in Figure \ref{fig:TPCFig1}. The U and V planes consist of induction wires which see a bipolar signal as an electron passes by. The Y plane consists of collection wires which see a unipolar signal as the electrons are collected on it. These three readings can be combined to determine the trajectory of the lepton and it's properties such as energy and momentum which can tell us about the neutrinos. However we must also know the time at which the interaction happens. The produced photons may move off to hit a photodetector, but there are not enough of them to signal that they were the products of the interaction. Fortunately, Liquid Argon is a good scintillator - the photons can hit an Argon nucleus, exciting an electron within, which will shortly decay, releasing more photons, which repeats enough photons are released to give a time for the interaction. Acrylic bars coated with Tetra-N-phenylbenzidine (TPB) or doped in bulk will be installed in the APA frame. Signals are then read out electronically to be used as a trigger for the interaction time in the Data Acquisition (DAQ). TPCs will be stacked back to back so that their APAs overlap. This will improve the chances of pile-up, where a large number of interactions occur within one event and physically close to each other in the detector meaning that there is an ambiguity to which interaction occurs before another one, but makes the construction more cost effective.
\begin{figure} [H]
\begin{center}
%\includegraphics[width=0.4\textwidth]{TPCFig1.png}
\caption{Figure showing the TPC and APA design. \cite{LBNETPCppt} \label{fig:TPCFig1}}
\end{center}
\end{figure}
\noindent The amount of neutrinos detected will depend on the supernovae type and distance from Earth. For a core-collapse event $\SI{10}{kPc}$ away from Earth, there will be of the order of a few hundred interactions per event (the smallest unit of time measured by the DAQ, which is $\SI{0.5}{\micro s}$) during the initial burst. This is orders of magnitude greater than the number of interactions from background sources, so would be immediately apparent when the DAQ is viewed. However the DAQ receives about $\SI{4.6}{Tb s^{-1}}$ of data which must be overwritten shortly after, meaning that important data such as a supernovae readout must be saved to memory. This is why we must construct a trigger which will know when a supernovae occurs and save the readout. To build this trigger, we must have information on some of the properties of supernovae neutrinos to be able to distinguish them from the background such as the number of interactions, their energy, momentum and many other properties of the interaction. Since real data will not be collected for a number of years, Monte Carlo (MC) simulations are used. MC simulations are a set of computational algorithms which give numerical results based on random sampling about a probability distribution. For example, say we expect the number of neutrino interactions (which we can call the number of MC truths - named truths since the simulation, these 'truths' are what are 'actually' happening) per event from a supernovae to be distributed according to a Poisson distribution.

\section*{\fontsize{11}{11}\selectfont Required software packages}
SNOwGLoBEs (Sudbury Neutrino Observatory General Long Baseline Experiment Simulator) is a software package (currently under development) which aims to simulate a supernova neutrino event occuring within a detector. SNOwGLoBEs computes the neutrino flux from a supernova under a given model of a supernova for a given flux model. The three models currently available are the Livermore model, the GVKM model and (most recently) the Garching model. The Garching model is the most relevant as it includes neutrino interactions within the supernova such as nucleon Bremsstrahlung, neutrino pair processes, weak magnetism and nucleon recoils, making it the best model of nuetrinos withing a supernova to date. After the flux has been calculated (and the resultant flux that an Earth-based detector 10 kpc from a supernova would recieve), SNOwGLoBEs then computes the cross-sections for interactions within a detector and uses smearing matricies to calculate the interaction product spectra and a detector response. Finally, detector efficiencies are calculated. 
\vspace{0.5cm}
\\SNOwGLoBEs required the software 'GLoBEs' to run. This can be downloaded from github and then using the \verb|'./configure'| followed by the commands \verb|'make'| and \verb|'make install'|, within the GLoBEs directory. However, one must ensure that the GSL (GNU Scientific Libraries) are included within the \verb|'PATH'| variable, otherwise one will have to specifiy when using \verb|'./configure'| the location of these libraries. SNOwGLoBEs can also be installed from github and installed in a similar fashion, although one must declare some environment variables to the GLoBEs installation files and to where SNOwGLoBEs is to be installed. One uses SNOwGLoBEs by generating .dat files containing flux information for a given model, compiling a 'pinched' file which is used to create the outputted .dat files whose data can be viewed using ROOT to produce histograms. More detailed instructions are available at \verb|'https://epp.phys.susx.ac.uk/SnoPlus/SuperNova/SnowGlobes'|.
\vspace{0.5cm}
\\ROOT is an object oriented framework for large scale data analysis, which utilises C++. It is capable of producing histrograms and other plots and has many in-built functions for optimising plots. A key feature of ROOT is that it utilises a data container called a tree, with its substructures branches and leaves, which are used to store data efficiently. ROOT has many depandancies, including gcc libraries. When attempting to run SNOwGLoBEs through ROOT on the local university run server, I had to ensure that the latest version of the gcc libraries were loaded in before sourcing ROOT.
\vspace{0.5cm}
\\The majority of my work on this project would be done on the Fermilab servers, thus one required a Fermilab account. One can make one following the instructions at \url{https://web.fnal.gov/collaboration/DUNE/SitePages/Getting%20Computer%20Accounts%20at%20Fermilab.aspx}. Since I already had an account, I renewed it for another year at \url{https://fermi.service-now.com/renew_acct_request.do}, although I also needed another password which can be done by calling \verb|9-00-630-840-2345|.
One will obtain their own directory on the Fermilab servers and the ability to access them. Fermilab uses Kerberos for strong authentication - one will have a kerberos pricipal and password. I would use this to login to the servers with the use of kinit. 
\vspace{0.5cm}
\\LArSoft (Liquid Argon Software) is a collaboration of software used which is used to simulate events within LArTPCs, which utilisies many libraries from ART (a collection of libraries useful for detectors). One can build a release of LArSOFT by following the instructions at \url{https://cdcvs.fnal.gov/redmine/projects/35ton/wiki/Getting_Started_Examples}, remembering to move to the feauture branch to be used (here 'feature/mbaird42/SupernovaAna') using \verb|git checkout|. When logging into a DUNE server, start-up scripts which source generic environment variables and commands used for LArSOFT, are automatically run. These can can be seen in my home directory \verb|'/dune/app/users/soughton'|. One may have to occasionally rebuild their LArSOFT release with a newer version following the same steps. One can build ones own code into the LArSoft repository using the \verb|'mrb i'| command. Once this is done, one can run their code over a file containing Monte Carlo truth and hit data using the 'fhicl' job file associated with ones code. For example I would run over a file using \verb|'lar -c supernovaana_job.fcl <filename>'|. This will output a .root file containing any histograms or N-Tuples created by the code which can then be viewed using ROOT.

\section*{\fontsize{11}{11}\selectfont Results from SNOwGLoBEs}
I assumed that supernovae neutrinos would follow the Garching flux model. This model was developed by running a Monte Carlo simulation to determine values for the average energies and fitting parameter $\alpha$ to produce flux values under the following relation
\begin{equation}
\phi(E_{\nu}) = N \left( \frac{E_{\nu}}{<E_{\nu}>} \right)^\alpha \exp \left( -(\alpha + 1) \frac{E_{\nu}}{<E_{\nu}>} \right),
\end{equation}•
\\where $N$ is a normalisation constant. Under the Garching model, SNOwGLoBEs was used to produce a time evolution plot in terms of neutrino luminosity, average energy and $\alpha$ as seen in Figure  \ref{fig:SnowglobesPlot1}. This plot reaffirmed what was expected about the energies and the events per bin for a supernova event and so I could proceed with the knowledge of what characteristics a trigger should look for.
\begin{figure} [H]
\begin{center}
%\includegraphics[width=0.4\textwidth]{SnowglobesPlot1.png}
\caption{Figure showing the time evolution of a supernova under the Garching model.  \label{fig:SnowglobesPlot1}}
\end{center}
\end{figure}


\section*{\fontsize{11}{11}\selectfont Investigation of hits within the detector}
My work would focus on finding a correlation between the positions of reconstructed hits within the detector from a supernovae event. I continued to edit my supernova analyser module code that I was working on in the summer of 2016. The module gets truth and hit information from the event and then fills an N-Tuple. Once having built and updated version of this code into the LArSoft repository I was using, I could run this analyser code over files containing truth and hit information for supernova neutrinos and their products, background radiological noise and electrical noise or any combination thereof. 
\vspace{0.5cm}
\\I needed to obtain information on the hits within the detector. A hit is counted as when an electron hits the APA plane, but its location and time is reconstructed to be where the electron is initially created - where it was ionised from an atom or the location of the neutrino interaction for the original lepton. To do this an ART module named 'BackTracker' which could cheat and find exactly the hit information (position, energy, momentum, particle type etc.) by looking into the Monte Carlo events - which could not be done for real data but is useful for building a picture of the detector. BackTracker proved a difficult module to use, LArSoft failed to recognise one of its functions 'HitToSimIDEs'. Rebuilding LArSoft to a later release did not resolve this issue nor did updating the complier that LArSoft uses. The issue was fixed by adding the relevant file name to the file that LArSoft requires to build code. I ran the analyser module mostly over a file containing information for a supernova event as well as for background radiological sources and managed to produce histograms for this file. 
\vspace{0.5cm}
\\A frequency histogram of the $x$-position of hits (the electrons drift in the $x$ direction towards the APA) shows a Gaussian-like, but non-Gaussian distribution of electrons as seen in Figure \ref{fig:xHist1}, whilst the histograms for the $y$ and $z$ hit positions were uniformly distributed. Note the lack of hits exactly at $x=0$ due to the presence of the beam around which the wires are wrapped preventing hits occuring there. It was hypothesised that this hit distribution was due electrons losing energy as they drift towards the APA and so any electrons which lose enough energy to re-ionise and are not seen as hits. If there is a non-uniform distribution of electron energies (the distribution is expected to be related to the Landau distribution and from a two-body decay distribution), then one would find that (assuming a loss of energies that is linear in $x$), that this energy distribution is effectively truncated by the time electrons reach the APA plane. To test this hypothesis I made a so-called 'T-Profile' which gave the average electron energy per $x$-bin. If the hypothesis was correct then the average energy would be higher further away from the APA plane as only the higher energy electrons would be able to reach the APA plane and be read as a hit. A plot of this Figure \ref{fig:TProfile1} shows this relationship.
\begin{figure} [H]
\begin{center}
%\includegraphics[width=0.4\textwidth]{xHist1.png}
\caption{Figure showing the frequency historgram for hits along the $x$-axis.  \label{fig:xHist1}}
\end{center}
\end{figure}
\begin{figure} [H]
\begin{center}
%\includegraphics[width=0.4\textwidth]{TProfile1.png}
\caption{Figure showing the average electron energy per $x$-bin. \label{fig:TProfile1}}
\end{center}
\end{figure}
\vspace{0.5cm}
A useful plot is a histogram of the number of hits in the $x$-$z$ cross-section, weighted by energy as shown for one event in Figure \ref{fig:xzCrossSection1event} and for one hundred events in Figure \ref{fig:xzCrossSection100events}. Here we see that the clustering of hits is much denser closer to the $x$ origin. 
\begin{figure} [H]
\begin{center}
%\includegraphics[width=0.4\textwidth]{xzCrossSection1event.png}
\caption{Figure showing the energy-weighted number of hits in the $x$-$z$ cross-section for one event.  \label{fig:xzCrossSection1event}}
\end{center}
\end{figure}
\begin{figure} [H]
\begin{center}
%\includegraphics[width=0.4\textwidth]{xzCrossSection100events.png}
\caption{Figure showing the energy-weighted number of hits in the $x$-$z$ cross-section for one hundred event.s \label{fig:xzCrossSection100events}}
\end{center}
\end{figure}

\section*{\fontsize{11}{11}\selectfont Future work}
A lepton produced from a neutrino interaction will have sufficient energy to ionise a large number of electrons which should show as a cluster of hits with a higher density of hits than would otherwise occur. Therefore if an algorithm can determine some measure of hit density and return a larger than average value, this should be a good indicator of a neutrino event. This would be relatively trivial to accomplish if there was no background noise - the difference between two vectors could find the maximum distance between hits within a cluster and find the number of hits within a sphere of this radius, or an ellipsoid or cone shape if the algorithm were advanced enough. However this becomes more complicated with noise, one would need a method of first determining a possible region of interest and then running the above algorithm.


\section*{\fontsize{11}{11}\selectfont Summary}
SNOwGLoBEs was used to confirm the expected energy and event time evolutions that would be important for building a supernova trigger. I have been able to obtain useful Monte Carlo truth and hit information about a neutrino interaction and produce useful histograms. The hypothesis of the absence of hits due to re-ionisation was tested and shown to be true (although work should be done to test whether the loss of hits strictly follows from the expected energy distribution mathematically). The next goal will to be to develop an algorithm that can determine whether a cluster of hits was caused by a supernova event.

\newpage
\section*{\fontsize{11}{11}\selectfont My Experiences}
During my project gained further experience in the programs and software I was be using. I also had to read-up on Liquid Argon detectors and supernovae neutrinos. I thoroughly enjoyed working on this project. My understanding of particle physics and of how experiments work has improved greatly. I have also gained experience of working with C++, SNOwGLoBEs, ROOT, LArSOFT and ART and of being in a real research group. I found the project to be challenging, but I liked that aspect as I was able to solve problems that I would not have otherwise faced during my studies.

\section*{\fontsize{11}{11}\selectfont Biography}
I have just completed my third year at Sussex, on the MPhys with Research Placement course. In the future, I would like to obtain a PhD and pursue a career in research. My areas of interest are particle physics and early-universe cosmology.


\newpage
\section*{\fontsize{11}{11}\selectfont Bibliography}
\begin{thebibliography}{20}

\bibitem{LBNEVol1} 
	R. Acciarri \textit{et at.} (2016),
	\emph{Long-Baseline Neutrino Facility (LBNF) and
Deep Underground Neutrino Experiment (DUNE)}.
	arXiv:1601.05471
		 
\bibitem{Scholberg} 
	K.Scholberg (2012),
	\emph{Supernova Neutrino Detection}.
	Department of Physics, 
	Duke University, Durham NC 27708, USA,
	
\bibitem{LBNEVol4} 
	R. Acciarri \textit{et at.} (2016),
	\emph{Annex 4A: The LBNE Design for a Deep
Underground Single-Phase Liquid Argon TPC}.
	 arXiv:1601.02984
	
\bibitem{LBNETPCppt} 
	B.Viren,
	\emph{Wire Cell Event Reconstruction Software
for LArTPC Detectors}.
	PowerPoint Presentation,
	Department of Physics,
	Brookhaven National Laboratory,
	2015	

\end{thebibliography}

\end{document}
